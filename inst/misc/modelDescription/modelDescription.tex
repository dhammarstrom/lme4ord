\documentclass{article}
\usepackage{amsmath}
\usepackage{bm}
\usepackage{fullpage}
\usepackage{mathrsfs}

\DeclareMathAlphabet{\mathpzc}{OT1}{pzc}{m}{it}

\newcommand{\mat}{\bm}
\newcommand{\trans}{^\top}
\newcommand{\vc}{\mathrm{vec}}
\newcommand{\dnorm}{\mathcal{N}}
\newcommand{\dexpfam}{\mathcal{D}}

\title{Extending the generalized linear mixed model for community ecologists}
\author{Steve Walker and Ben Bolker}
\date{}

\begin{document}

\maketitle

\section{Introduction}

Data on ecological communities are very difficult to model.  This
difficulty arises 
Community data are multivariate, which does not in and of itself
indicate data modelling challenges, It is multivariate and non-normal,
phylogenetically, spatially, and temporally auto-correlated.

% \begin{equation}
%   \label{eq:1}
%   \mat Z = \begin{bmatrix}
%     \mat U\trans\otimes \mat I_n \\
%     \mat 1_m\trans\otimes\mat I_n \\
%     \mat I_m \otimes\mat 1_n\trans \\
%   \end{bmatrix}\trans
% \end{equation}

% \begin{equation}
%   \label{eq:2}
%   \mat Y = \mat X\mat\beta + \mat b_n\mat 1_m\trans + \mat 1_n\mat b_m\trans + \mat U\mat V\trans
% \end{equation}

\section{The model}

\subsection{The standard generalized linear mixed model}

The generalized linear mixed model takes the form,
\begin{equation}
  \label{eq:6}
  \bm\eta = \bm X\bm\beta + \bm Z\bm b
\end{equation}
\begin{equation}
  \label{eq:10}
  \bm b = \bm\Lambda_{\bm\theta}\bm u
\end{equation}
\begin{equation}
  \label{eq:7}
  \bm u \sim \dnorm(0, \bm I)
\end{equation}
\begin{equation}
  \label{eq:8}
  \bm y \sim \dexpfam (\bm\eta, \bm\phi)
\end{equation}
where ...  

\subsection{The extended model}

GLMMs are suitable for modelling a wide variety of data.  However, in
community ecology the response variables are often species abundances
or species presence-absence, and such data are often characterized by
correlations between species even after the effects of environmental,
phylogenetic, and space are accounted for.  These correlations among
species are typically due to either unmeasured site and species
characteristics and species interactions.  These correlations can be
accounted for by allowing $\bm Z$ to depend on a vector of parameters,
$\bm\psi$.  Therefore the model that we will be using is given by,
\begin{equation}
  \label{eq:6}
  \bm\eta = \bm X\bm\beta + \bm Z_{\bm\psi}\bm\Lambda_{\bm\theta}\bm u
\end{equation}


\end{document}
